Pursuing a doctoral degree during a global pandemic has been a challenging yet incredible and rewarding journey of personal growth and intellectual stimulation. This dissertation would not have been possible without the guidance and support of many incredible people around me.

First and foremost, I am deeply grateful to my advisor, Dr. Kamen Kozarev. Thank you for giving me the opportunity to work on such an interesting and exciting project (SPREADFAST: Solar Particle Radiation Environment Analysis and Forecasting—Acceleration and Scattering Transport). Thank you for granting me the flexibility to work and the freedom to explore and experiment with new ideas. Your insightful feedback, unwavering encouragement, and dedication to my success have been invaluable throughout this process. I am incredibly grateful for your patience during moments of doubt and your enthusiasm that reignited my passion when I needed it most, as well as your ability to help me refine my thoughts and ideas. I am truly fortunate to have had you as my advisor.

I extend my heartfelt gratitude to my colleagues in the Sun and Solar System Department at the Institute of Astronomy and National Astronomical Observatory, Bulgarian Academy of Sciences especially Dr. Rositsa Miteva, Dr. Oleg Stepanyuk, and Dr. Momchil Dechev. Thank you for the stimulating discussions, the collaborative spirit, and for generously sharing your expertise. Your valuable advice, constructive feedback, and support have greatly enriched my experience. I am also grateful to Nestor Arsenov and Yordan Darakchiev for the insightful discussions on deep learning models, which have been valuable in shaping my understanding.

A special thank you goes to Dr. Pietro Zucca and Dr. Peijin Zhang for introducing me to the fascinating world of radio observations from the Low-Frequency Array (LOFAR) and guiding me through the intricacies of working with the data and relevant software. Thank you for the cherished moments we shared during my visits to the Netherlands and other places; those experiences have left an indelible imprint.

Furthermore, I am grateful for the financial support provided by the MOSAIICS project (Modeling and Observational Integrated Investigations of Coronal Solar Eruptions) and the STELLAR project (Scientific and Technological Excellence by Leveraging LOFAR Advancements in Radio Astronomy). This funding has been a tremendous blessing, allowing me to focus wholeheartedly on my research without financial constraints, and has significantly contributed to the successful completion of this dissertation.

My deepest appreciation goes to my family. Thank you for your unwavering love, belief in me, and constant prayers and encouragement throughout this long journey. You have been my source of strength and motivation.
To my dear friends, thank you for the countless moments of laughter, the much-needed distractions, and for reminding me of the vibrant world beyond academia. Your support and companionship, especially during my initial period in Bulgaria, have rendered this experience all the more enjoyable.

Finally, I would like to acknowledge and express my heartfelt gratitude to all those who have directly or indirectly contributed to this work. The collective efforts of all contributors, no matter how big or small, have played a crucial role in bringing me to this point and the successful completion of this dissertation.
Those four years were the most beautiful time of my life and I will always cherish them. Thank you all from the bottom of my heart.