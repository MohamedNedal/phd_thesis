Here, I provide definitions for key domain-specific terms and measurement concepts used consistently throughout the thesis. Additionally, relevant terminology will be introduced within the corresponding chapters. Below is a compilation of the essential technical terms and acronyms featured in this work:

\vspace{0.5cm}

CME -- Coronal Mass Ejection

ICME -- Interplanetary Coronal Mass Ejection

SF -- Solar Flare

CIR -- Corotating Interaction Region

IMF -- Interplanetary Magnetic Field

GS -- Geomagnetic Storm

Dst -- Disturbance storm time

AU -- Astronomical Unit

MPA -- Measurement Position Angle

SEPs -- Solar Energetic Particles/Protons

ESP -- Energetic Storm Particle

EUV -- Extreme Ultra-Violet

CBF -- Coronal Bright Front

SRB -- Solar Radio Burst

DH -- Decameter-Hectometric

SOHO -- Solar and Heliospheric Observatory

LASCO -- Large Angle and Spectrometric Coronagraph

ERNE -- Energetic and Relativistic Nuclei and Electron

EIT -- Extreme ultraviolet Imaging Telescope

TRACE -- Transition Region and Coronal Explorer

ESA -- European Space Agency

MHD -- Magneto-Hydro-Dynamic

MAS -- Magnetohydrodynamic Algorithm outside a Sphere

PSI -- Predictive Science Inc.

PFSS -- Potential Field Source Surface

AIA -- Atmospheric Imaging Assembly

SDO -- Solar Dynamic Observatory

EUVI -- Extreme Ultraviolet Imager

STEREO -- Solar Terrestrial Relations Observatory

LOFAR -- Low-Frequency Array

PSP -- Parker Solar Probe

GOES -- Geostationary Operational Environmental Satellite

SN -- Sunspot Number

SC -- Solar Cycle

sfu -- solar flux units

pfu -- proton flux units

L1 -- First Lagrange point

SPDF -- Space Physics Data Facility

SILSO -- Sunspot Index and Long-term Solar Observations

NOAA -- National Oceanic and Atmospheric Administration

NASA -- National Aeronautics and Space Administration

GSFC -- Goddard Space Flight Center

SPREAdFAST -- Solar Particle Radiation Environment Analysis and Forecasting–Acceleration and Scattering Transport

CASHeW -- Coronal Analysis of SHocks and Waves

DSA -- Diffusive Shock Acceleration

SDA -- Shock Drift Acceleration

S2M -- Synthetic Shock Model

EPREM -- Energetic Particle Radiation Environment Module

DEM -- Differential Emission Measure

FLCT -- Fourier Local Correlation Tracking

CM -- Centers of Mass

GC -- Geometric Center

GCS -- Graduated Cylindrical Shell

NN -- Neural Network

ML -- Machine Learning

DL -- Deep Learning

CNN -- Convolutional Neural Network

GAN -- Generative Adversarial Networks

RNN -- Recurrent Neural Network

BiLSTM -- Bi-directional Long short-term Memory

Adam -- Adaptive moment estimation

MIMO -- Multi-Input Multiple Output

MSE -- Mean Squared Error

MAE -- Mean Absolute Error

MSLE -- Mean Squared Logarithmic Error

TP -- True Positive

TN -- True Negative

FP -- False Positive

FN -- False Negative

POD -- Probability of Detection

POFD -- Probability of False Detection

FAR -- False Alarm Rate

CSI -- Critical Success Index

TSS -- True Skill Statistic

HSS -- Heidke Skill Score