\chapter{Characterization of Coronal Bright Fronts}
\label{chapter2}

\section{Introduction}

\section{Observations}

\section{Data Analysis and Methods}

\section{Results and Discussion}

\section{Conclusions}






\section{Wavetrack: A Flexible Framework for Automated Recognition and Tracking of Wave-Like Events in Solar Imagery}
\subsection{Overview}
Solar eruptive events involve complex phenomena like flares, filament eruptions, coronal mass ejections (CMEs), and CME-driven shock waves. CME-driven shocks in the corona and interplanetary space are considered the primary accelerators of solar energetic particles (SEPs), mainly through diffusive shock acceleration (DSA) and shock drift acceleration (SDA) processes \citep{reames_2021}. Substantial efforts have characterized and modeled SEP acceleration under ideal conditions \citep{vainio_2008, sokolov_2009, kozarev_2013}. Recent work has focused on building more realistic CME-shock and SEP acceleration models using observational data \citep{vourlidas_2012, kwon_2014, kozarev_2015, kozarev_2019}.

Knowledge of CME-shock interactions with 3D coronal fields is critical for understanding SEP acceleration efficiency and spread \cite{rouillard_2016}. DSA modeling requires deducing shock front shapes from observations since the local magnetic field-shock angle strongly affects acceleration \cite{guo_2013}. CMEs often over-expand laterally early on, altering the overall compressive front \cite{bein_2011, temmer_2016}. Thus, idealized shock surface descriptions used in modeling should be improved \citep{vourlidas_2012, kwon_2014, rouillard_2016}.

EUV imaging enables shock characterization, using instruments like STEREO/EUVI \citep{wuelser_2004} and SDO/AIA \citep{lemen_2012}. Their resolution and multi-wavelength coverage allow time-dependent modeling of SEP acceleration \citep{kozarev_2016, kozarev_2017, kozarev_2019}. Automated solar feature detection has increased due to Big Data demand. Fundamental techniques are reviewed in \citet{aschwanden_2010} and applied for features at different heights \citep{perez_Suarez_2011}. Some detect sunspots and active regions \citep{curto_2008}.

EUV wave recognition and tracking are challenging due to their weak intensity near other features. Algorithms exist but can be complex \citep{podladchikova_2005, verbeeck_2014, long_2014, ireland_2019}. CorPITA \citep{long_2014} and AWARE \citep{ireland_2019} fit shapes to flare-centered wave sectors, but use different images or persistence filters that introduce artifacts. They currently only work on the disk. 

This paper presents Wavetrack - a flexible, object-oriented Python library for general solar feature detection and tracking. At its core combines multiscale representation \citep{starck_2002}, \'A trous wavelets \citep{akansu_1991, holschneider_1989}, and filtering. Wavetrack can generate training sets for machine learning-based recognition.

Machine learning is increasingly applied in solar physics, like for EUV irradiance mapping \citep{szenicer_2019}, solar flare prediction \citep{li_2013}, and magnetic flux generation \citep{kim_2019}. However, data-driven CME tracking is limited by small training sets. Wavetrack can facilitate these by converting results into annotated sets.

At the core of Wavetrack is a general feature recognition and tracking method. It is implemented as an easy-to-integrate, open-source Python library. The modular structure allows configuring applications by tuning a few threshold parameters. The calculation scheme combines multiscale data representation \citep{starck_2002} and the \'A trous wavelet transform \citep{akansu_1991, holschneider_1989}. This is supplemented by image filtering techniques for noise removal and edge enhancement.

Wavetrack moves beyond complex existing algorithms that target specific phenomena. It provides a flexible framework for detecting various solar features in EUV, white light, and other observations. The modular design enables training set generation for machine learning approaches. Wavetrack facilitates automated analysis to uncover new knowledge from the wealth of solar data.

The \'A trous wavelet transform enables isolating features at different scales. This allows separating noise from the signal and enhancing edges. The wavelet coefficients encode the multiscale structure efficiently. Thresholding coefficients suppress noise by zeroing insignificant values.

The filtering techniques refine the signal further. Morphological operations like opening remove artifacts and smooth edges. Contrast enhancement sharpens edges and boundaries. The result highlights features while suppressing noise. 

Wavetrack takes advantage of these techniques to recognize faint features like EUV waves. It tracks them by identifying significant coefficients across wavelet scales. This avoids pre-processing like difference images that can introduce false signals.

The modular structure makes Wavetrack adaptable. Users can build specific applications by setting parameters like thresholds and filters. The original data is not altered, only the processed copies are used for feature detection. This prevents information loss.

Wavetrack will be open-source to facilitate community involvement. Users can contribute modules, optimizations, and new techniques. This collective effort will enhance solar data analysis capabilities for the Big Data era. Together we can uncover new knowledge to better understand our dynamic Sun.




Some techniques used for image processing and analysis in the context of solar observations include:
1. Wavelet Transform: The \'A trous wavelet transform is a multi-scale data representation concept that is commonly used in solar image analysis. It allows for the extraction of features at different decomposition and intensity levels.
2. Image Filtering: Various image filtering techniques are employed to enhance and extract specific features in solar images. These techniques can help in detecting and tracking phenomena such as CME shock waves and filaments.
3. Feature Detection Algorithms: Automated algorithms are used to detect and identify solar features in different types of observations. These algorithms utilize image processing techniques to identify features like sunspot groups, active regions, and eruptive fronts.
4. Machine Learning: Machine learning methods, such as Convolutional Neural Networks (CNNs) and Generative Adversarial Networks (GANs), are increasingly being applied in solar image analysis. These techniques can be used for tasks like predicting solar flares or generating magnetic flux distributions.
5. Solar Feature Tracking: Tracking the evolution of solar features over time is an important aspect of image analysis. Tracking algorithms can be used to follow the movement and changes in features like CMEs, filaments, and EUV waves.
These techniques and algorithms contribute to the understanding of solar phenomena and can aid in the development of models for solar activity and space weather forecasting.




















The study utilized observations from the Atmospheric Imaging Assembly (AIA) instrument on the Solar Dynamics Observatory (SDO). The AIA instrument is an EUV imager that provides high-resolution and high-temporal observations of the Sun. The AIA instrument is particularly useful for detecting and characterizing large-scale shocks, known as EUV waves or Coronal Bright Fronts (CBFs).
The study specifically focused on the AIA channels centered on 193Å and 211Å wavelengths. The 193Å data was obtained using a standard AIA pipeline with the SunPy package. Base images were created by averaging a series of images prior to the eruption. Base difference images were then constructed by subtracting the base images from the current raw image of the sequence. This allowed for the enhancement of the change in intensity caused by CBFs, while omitting static details and reducing noise.
The CBFs observed in EUV are usually dim features that exist in a narrow part of the wide dynamic range of the raw image data. To overcome the challenge of recognizing and tracking these dim objects, the study pre-selected the part of the dynamic range where the shock wave is revealed from the base difference image of the sequence.
Overall, the SDO/AIA telescope observations were utilized to study and track the CBFs and their evolution over time. The observations provided the necessary data to analyze the spatial and temporal relationships of the CBFs and other solar dynamic features.

The specific image filtering techniques used in the method include thresholding, wavelet decomposition, and segmentation.
First, thresholding is applied to the absolute values of the pixels in the base difference images. This helps narrow the dynamic range of the image and focus on the segment where the object of interest is located.
Next, the base difference images are decomposed using the \'A trous wavelet technique into a series of scales. Each wavelet coefficient is then subjected to relative thresholding based on the statistical distribution of the pixel intensities at each decomposition level.
After the segmentation process, object masks are obtained for each time step. These masks are then multiplied by the original data to reveal the intensity of different parts of the object.
Overall, the method utilizes a combination of thresholding, wavelet decomposition, and segmentation to filter and analyze the images.

The statistical distributions of pixel intensities in base and running difference images can vary. In the case of base difference images, the pixel intensities are narrowed down to a specific segment where the object of interest is located, such as a coronal bright feature (CBF). The absolute values of the threshold interval are selected to focus on this segment. On the other hand, running difference images are used to determine the moment when an eruption starts. The statistical distribution of pixel intensities in these images depends on the specific features and objects being observed.

The multi-scale data representation concept, specifically the \'A trous wavelet transform, contributes to the recognition and tracking of solar images by providing a hierarchical decomposition of the data. This allows for the extraction of certain objects and their masks from the imaging observations, enabling their tracking and evolution over time. The \'A trous wavelet decomposition enhances the clarity and intensity of faint features in solar images, such as EUV waves, making them easier to detect and track. This methodology provides a flexible framework for solar feature detection and can be applied to various types of solar phenomena observed in different wavelengths. It offers a valuable tool for analyzing and understanding the dynamics of solar features, such as CME shock waves and filaments.

The purpose of the Wavetrack package mentioned in the document is to provide a method for the automated detection and tracking of dynamic coronal features in solar observations. It utilizes wavelet decomposition, feature enhancement and filtering, and object recomposition to identify and track features such as coronal bright fronts (CBFs) and eruptive filaments. The package is designed to work with both on-disk and off-limb features and can track features that split into separate parts over time. It is a flexible, object-oriented framework written in Python and is freely available for download and use.

Wavetrack uses AIA 193Å images for tracking coronal waves.
Wavetrack tracks coronal waves by utilizing wavelet decomposition, feature enhancement, and filtering techniques. It applies an \'A trous wavelet decomposition method to the observational data, which enhances the clarity and intensity of faint features such as EUV waves. The algorithm convolves the image with different iterations of the wavelet filters, allowing it to capture different scales of features.
After the wavelet decomposition, Wavetrack performs automated feature recognition by applying intensity thresholding, image posterization, median filtering, image segmentation, and intensity level splitting methods. These techniques help identify and isolate the coronal wave features in the data.
The output of Wavetrack is a time-dependent pixel mask that represents the tracked coronal wave. This pixel mask can be applied to the original data to generate a final feature map. Wavetrack is capable of tracking both on-disk and off-limb features, and it can successfully track features that split into separate parts over time.
Overall, Wavetrack combines wavelet decomposition and automated feature recognition to effectively track coronal waves in solar observations.

To track filaments using Wavetrack, the following process is followed:
1. Wavelet Decomposition: The observational data is decomposed using wavelet decomposition. This helps in identifying different scales of features in the image.
2. Feature Recognition: The decomposed image is processed using various techniques such as intensity thresholding, image posterization, median filtering, and image segmentation. These techniques enhance the features of interest, making them easier to track.
3. Object Mask Generation: The processed image is used to generate time-dependent feature pixel masks. These masks isolate the filament features in the image.
4. Image Recomposition: The image can be recomposed from the weighted wavelet scales after applying filtering techniques. This helps in obtaining a final feature map of the filaments.
The type of images utilized for filament tracking depends on the source data and the velocity of the filaments. In general, inverted AIA 193Å images are used for tracking filaments. These images provide better visibility of the filaments due to their distinct contours and lower dynamic range. However, depending on the specific event and the characteristics of the filaments, other types of images such as Base Difference, Running Difference, or Raw Intensity data may also be used.

The FLCT model uses the Fourier Local Correlation Tracking (FLCT) method to calculate the plane-of-sky velocity and speed of solar features. This method involves analyzing consecutive images and tracking the motion of the features over time. The FLCT algorithm applies the FLCT method to the object masks generated by the Wavetrack code, resulting in detailed maps of the instantaneous velocity of the solar feature of interest. The velocity is represented by arrows indicating the direction of motion, while the speed is represented by the length of the arrows or through color mapping. The FLCT model provides valuable insights into the expansion and evolution of solar features such as Coronal Bright Fronts (CBFs) and erupting filaments.

The document presents several examples and case studies that demonstrate the application of the Wavetrack package. Here are some of the examples mentioned:
1. May 11, 2011 event: The Wavetrack algorithm was used to track both an erupting filament and the coronal bright front (CBF) it drives. The relationship between the driver and wave was studied using the segmented features from AIA 193Å observations.
2. September 29, 2013 event: Wavetrack was applied to a large-scale filament in a slow eruption. The segmented filament feature was overlaid on the solar corona, and the on-disk feature was consistently tracked during the eruption.
3. December 12, 2013 event: Wavetrack was used to track the evolution of driven and non-driven regions of the CBF. The method revealed the relation between the CBF and eruptive features.
These examples demonstrate the versatility of Wavetrack in recognizing and tracking various solar features, both on the solar disk and off the solar limb.

For the June 07, 2011 event, the variation in speed between the geometric center (GC) and center of mass (CM) is up to 300 km/s in the radial direction. The angle between the GC and CM changes quite a lot as different parts of the compressive front brighten and dim. However, the angle remains relatively stable, changing only slightly during this event.

The main results of the paper are as follows:
1. The Wavetrack framework, which utilizes an \'A trous wavelet decomposition method, is effective in enhancing solar EUV and white light images and improving the clarity and intensity of faint features.
2. The Wavetrack method allows for the segmentation and tracking of solar dynamic features, such as coronal bright fronts (CBFs), over time. It provides a simple metric with a one-dimensional time series for characterizing the evolution of these features.
3. The Wavetrack framework is widely applicable to different types of solar dynamic features and different observational data.
4. The method has some limitations, such as the need for manual input in setting object criteria and fine-tuning parameters for specific events. These limitations will be addressed in future work.
Overall, the Wavetrack framework provides a valuable tool for analyzing and tracking solar dynamic features in observational data.

The conclusions drawn from the study are as follows:
1. The Wavetrack method is effective in segmenting and tracking solar dynamic features, such as coronal bright fronts (CBFs) and filaments, in observational data.
2. The Wavetrack method allows for the analysis of feature evolution over time by calculating the time-dependent vector between the pixel geometric center and the center of mass.
3. The method can successfully capture the extent of CBFs in consecutive time steps, even in regions with different pixel distributions and intensities.
4. Wavetrack is capable of tracking multiple separate parts of the same feature.
5. The method is widely applicable to different types of solar dynamic features and different observational data.
6. The limitations of the method, such as the need for manual input in setting object criteria and fine-tuning parameters for specific events, will be addressed in future work.





\section{Geomagnetic Storms: CME Speed De-Projection vs. In Situ Analysis}
This study, led by \citet{miteva_2023}, examines the relationship between the intensity of geomagnetic storms (GS) and parameters of solar and interplanetary phenomena. We utilize the recently developed PyThea framework to reconstruct the 3D geometry of geo-effective CMEs and compare on-sky and de-projected values, focusing on the reliability of the de-projection capabilities. Spheroid, ellipsoid, and graduated cylindrical shell models are used. We collected parameters of the GS-associated phenomena. Considerable variation in 3D de-projections of CME speeds was obtained depending on the reconstruction model chosen and subjective observations. Fast CME speeds combined with frontal magnetic structure orientation when reaching Earth's magnetosphere proved the best indicator of GS strength. More accurate estimations of geometry, direction, and de-projected speeds are critical for operational GS forecasting in space weather prediction schemes.

\subsection{Overview}
Solar eruptive events like coronal mass ejections (CMEs) and solar flares (SFs) can generate disturbances that propagate through interplanetary space and impact Earth's magnetosphere, leading to space weather effects \citep{fletcher_2011, webb_2012, klein_2017, temmer_2021}. The electromagnetic radiation from SFs arrives at Earth first, followed by energetic particles. The plasma clouds of CMEs take longer, from tens of hours up to a few days, to impact Earth's environment \citep{malandraki_2018, gopalswamy_sun_sw_2022}. The temporary disruption of Earth's magnetosphere and atmosphere due to these solar events are known as geomagnetic storms (GSs) \citep{gonzalez_1994, saiz_2013, lakhina_2016}.

The coupling between solar wind plasma and Earth's magnetosphere occurs through magnetic reconnection, enabled when the interplanetary (IP) magnetic field turns southward (negative Bz component) and impacts Earth at high speed, as during CMEs \citep{dungey_1961, akasofu_1981, echer_2022}. This leads to increased particle injection into the magnetosphere and atmosphere, causing bright auroral displays. Drifting electrons and protons also drive the westward ring current responsible for decreases in the horizontal magnetic field measured by the disturbance storm time (Dst) index \citep{gonzalez_1994, saiz_2013, lakhina_2016}.

Fast CMEs propagating through IP space (ICMEs) cause the most intense GSs, with sudden Dst decreases, compared to gradual storms from corotating interaction regions (CIRs) \citep{tsurutani_1997, zhang_2007, wu_2016, borovsky_2006}. ICME shock waves and magnetic ejecta produce cascading effects in near-Earth space that can disrupt technology \citep{pulkkinen_2007}.

Earth-directed fast ejecta with strong southward magnetic fields inside are the most geoeffective. However, single spacecraft observations are limited by projection effects, leading to uncertain CME speed estimations \citep{paouris_2021, kouloumvakos_2022}. Previous studies found no clear relationship between GS intensity and solar flare parameters or CME properties like projected speed and angular width \citep{samwel_2023}. Furthermore, CME magnetic structure cannot be derived from remote sensing data. Reliable solar or near-Sun measurements that provide early warnings for potential GS strength remain lacking.

Accurately predicting when an incoming disturbance will impact Earth requires determining the arrival time and speed of CMEs. Different portions of these large structures, such as the apex or flanks, can hit Earth upon arrival at 1 AU. Flank hits may only involve the CME sheath while apex hits include both sheath and ejecta, leading to different magnetospheric effects \citep{kay_2018}. Therefore, deducing CME directionality and 3D geometry is important. To maximize forecast lead time, these parameters should be estimated as early as possible when the CME emerges in coronagraph fields of view. In images, CMEs appear as line-of-sight integrated brightness enhancements projected onto the observing plane \citep{vourlidas_2003, jackson_2010}.

Further developing reconstruction techniques to correct for projection effects can improve CME propagation understanding and impact forecasting \citep{thernisien_2009, mierla_2010, wood_2010, thernisien_2011}. Several CME propagation models exist \citep{odstrcil_2004, xie_2004, vrvsnak_2013, pomoell_2018}. A study found 2D CME speeds underestimate 3D speeds by \almost20\% while 2D widths overestimate 3D widths \citep{jang_2016}. Another study showed observer bias in 3D modeling using graduated cylindrical shells, even for experienced observers \citep{verbeke_2022}. CME structure interpretation differs, and line-of-sight integration leads to non-unique solutions.

This study focuses on deducing CME directionality and near-Sun speeds using new tools like PyThea \citep{kouloumvakos_2022}. We analyze geo-effective cycle 24 CMEs using multiple reconstruction techniques by two team members. Comparisons are made between derived parameters and storm intensity (Dst). CME speed correlations with ICME and IP shock speeds evaluate 3D de-projection value for arrival forecasting. Other IP parameters are also examined, including shock speed, plasma jumps, and magnetic fields near L1. We examine the correlation between the intensity of geomagnetic storms and parameters of solar and interplanetary phenomena. More specifically, we focus on the speed and geometry of CMEs, as well as interplanetary shocks, as these factors are known to play a significant role in the occurrence and intensity of geomagnetic storms. The objective of the study is to analyze the correlations between the intensity of geomagnetic storms (GS) and parameters of solar and interplanetary (IP) phenomena. The study also aims to perform 3D geometry reconstructions of geo-effective coronal mass ejections (CMEs) using the PyThea framework and compare the de-projected CME speeds with in situ data.

\subsection{Data Analysis and Methods}
The event selection process for this investigation commenced with the identification of significant Geomagnetic Storms (GSs) within Solar Cycle 24 (SC24). These storms were characterized by a Dst index exceeding 100 nT, as per the classification outlined in [7]. A total of 25 GSs were identified, with Dst indices ranging from 101 to 234 nT. Previous studies have explored the solar and Interplanetary (IP) origins of these GSs in SC24 [34–37]. However, the comprehensive review of all pertinent literature falls beyond the purview of this study. It is noteworthy that SC24 exhibited a reduced number of GSs in comparison to preceding solar cycles [38].

In our investigation, distinct from prior analyses, our objective was to establish a causal connection between the identified GSs and potential IP and/or solar phenomena. This approach mirrors methodologies employed by other researchers [14,39–42]. To delineate the solar and IP drivers, we employed an association procedure widely acknowledged in the field of Space Weather (SW) research. The methodology involves searching for the IP and solar origins of a GS storm within a specific timeframe preceding the reported GS occurrence at Earth. The sequential steps of our approach are delineated below:

1. **Temporal Association with IP and ICME Events:**
We initiated the analysis with a temporal association between the GS and recorded IP shocks in the vicinity of Earth. This association was established within a 1-day period preceding the reported minimum Dst of the GS. A parallel procedure was employed for the association with Interplanetary Coronal Mass Ejections (ICMEs) reported in proximity to Earth. Additionally, animations from http://helioweather.net/archive/ (accessed on 5 April 2023) were utilized to validate potential ICME and IP shock candidates.

2. **Association with Coronal Mass Ejections (CMEs):**
Subsequently, we extended the association to include Coronal Mass Ejections (CMEs) within a 3-to-5 day window prior to the IP (or GS) timing. Information from available solar and IP event catalogs, as well as animations from http://helioweather.net/archive/ (accessed on 5 April 2023), was employed for this purpose.

3. **Completion of the Association with Solar Flares (SFs):**
The final step involved associating the GS with the identification of a Solar Flare (SF) linked to the previously associated CME. This association was based on timing (within one hour between the SF onset and CME timing) and location constraints (the SF location had to align with the solar quadrant reported in the measurement position angle, MPA, of the CME).

All utilized databases, catalogs, and publicly available lists pertinent to the analysis are summarized as follows:

- GS database (Kyoto): [https://wdc.kugi.kyoto-u.ac.jp/dstdir/index.html](https://wdc.kugi.kyoto-u.ac.jp/dstdir/index.html) (accessed on 5 April 2023)
- SF database (GOES): [http://ftp.swpc.noaa.gov/pub/warehouse/](http://ftp.swpc.noaa.gov/pub/warehouse/) (accessed on 5 April 2023)
- CME catalog (SOHO-LASCO): [https://cdaw.gsfc.nasa.gov/CME_list/](https://cdaw.gsfc.nasa.gov/CME_list/) (accessed on 5 April 2023)
- ICME database: [https://wind.nasa.gov/ICME_catalog/ICME_catalog_viewer.php](https://wind.nasa.gov/ICME_catalog/ICME_catalog_viewer.php) (accessed on 5 April 2023) (Wind); [https://izw1.caltech.edu/ACE/ASC/DATA/level3/icmetable2.htm](https://izw1.caltech.edu/ACE/ASC/DATA/level3/icmetable2.htm) (accessed on 5 April 2023) (ACE)
- IP shock database (Wind): [http://www.ipshocks.fi/database](http://www.ipshocks.fi/database) (accessed on 5 April 2023); [https://lweb.cfa.harvard.edu/shocks/wi_data/](https://lweb.cfa.harvard.edu/shocks/wi_data/) (accessed on 5 April 2023)

\subsubsection{GSs and IP Phenomena}
The findings pertaining to Geomagnetic Storms (GSs) and their associated Interplanetary Coronal Mass Ejections (ICMEs) and Interplanetary (IP) shocks are succinctly presented in Table 1. The initial column designates the event number (#), a consistent reference utilized throughout this paper. Columns (2) and (3) detail the GS date, hour (mm-dd/hr), and the corresponding Dst index (in nT). Subsequently, columns (4)–(6) expound upon the ICME parameters [43], drawing from the Wind database accessible at https://wind.nasa.gov/ICME_catalog/ICME_catalog_viewer.php (accessed on 5 April 2023). The sheath duration (D, in hours), representing the temporal span between the initiation times of the ICME and the magnetic structure, is computed from the available timings within the plots offered by the aforementioned Wind database, and is presented in column (7). The ICME in situ measured speed is extracted from both the Wind and ACE databases, with the exception of E11 where no ICME is reported.

Column (8) integrates the minimum Bz component throughout the Interplanetary Coronal Mass Ejection (ICME) duration, as ascertained from the data available at https://cdaweb.gsfc.nasa.gov/ (accessed on 5 April 2023), thereby enhancing the comprehensiveness of the dataset. In Column (9), we conduct a qualitative assessment of the ICME arrival orientation. Specifically, we visually inspect the intersection point between the ICME structure and Earth using ecliptic plane animations accessible at http://helioweather.net/archive/ (accessed on 5 April 2023). The orientations are denoted as either 'hit' (nose) or 'f' (flank) arrivals. Instances of discrepancies among diverse data sources, such as the presence of solar wind streams or Corotating Interaction Regions (CIRs) visible in animations contradicting ICME arrivals identified through in situ data, are marked as 'u' (uncertain) in the same column. This classification signifies situations where a distinct ICME structure propagating through the Interplanetary (IP) space could not be conclusively observed. Noteworthy is the occurrence of swift solar wind flows and/or CIRs recorded near Earth during ICME and/or shock wave events. For instance, in the cases of E11 and E18, a CIR was identified as their IP origin by [35]; however, in contrast to these findings, our methodology does not differentiate between ICME and sheath origins.

The final columns, (10)–(13), enumerate the properties of the IP shock, incorporating timing, speed, magnetic field, density, temperature jump at the shock interface, and the Mach number (Mms). These details are sourced from Wind satellite data, http://www.ipshocks.fi/database (accessed on 5 April 2023), with the exception of E24, where the median shock speed is derived from https://lweb.cfa.harvard.edu/shocks/wi_data/ (accessed on 5 April 2023). Notably, for E17, E18, and E25, no IP shocks are reported in either database.

\subsubsection{GSs and Solar Phenomena}
In the context of five cases denoted as E05, E11, E17, E18, and E22, our attempts to identify Solar Flares (SFs) or Coronal Mass Ejections (CMEs) were unsuccessful. Furthermore, in an additional six cases, the specification of SFs proved unattainable. The ensuing presentation provides details on the parameters of the remaining cases, specifically focusing on the solar origins associated with Geomagnetic Storms (GSs), as outlined in Table 2.

Columns (2)–(5) of Table 2 elucidate the properties of the GS-associated SFs, while columns (6) to (9) furnish the parameters of the GS-associated CMEs. The identified SFs exhibit a range of magnitudes from C1.2 to X5.4 and are predominantly positioned proximate to the solar disk center, with the exception of E02 and E03. The CMEs associated with these events possess on-sky projected speeds, denoted as 2D, spanning from 126 to 2684 km/s, extracted from the SOHO-LASCO CDAW catalog. Notably, the majority of the GS-associated CMEs (15 out of 20) exhibit a halo configuration, while three others are in close proximity to halo.

It is imperative to note that events with uncertain CME origins have been excluded from the subsequent 3D analyses. Specifically, in the case of E07, the unique orientation of the double CME rendered the de-projection procedure unfeasible for the same CME structure, leading to its exclusion from the 3D analyses. Additionally, for seven other cases (E12, E14–E16, E19, E23, E25), the online tool utilized for analyses failed to retrieve data simultaneously from both spacecraft. Consequently, these cases have also been omitted from the 3D analyses.

For the remaining 12 cases, successful 3D CME speed reconstructions were achieved from each model. The mean values, based on 2 or 4 available fits (as detailed in the subsequent subsection), are presented in the concluding columns (10)–(12) of Table 2. This comprehensive overview provides a foundation for the subsequent analytical discussions, offering a detailed characterization of the solar events associated with the studied GSs.

\subsubsection{PyThea 3D De-Projection Tool}
The de-projection methodology employed in this investigation relies on the innovative PyThea online tool designed for the 3D reconstruction of Coronal Mass Ejections (CMEs) and shock waves [19]. All three available models within PyThea, namely spheroid, ellipsoid, and GCS, are applied in this study. The fitting procedure is conducted independently by two observers within our team. Figure 1 presents an illustrative example of the fits for event E03.

Upon scrutinizing the fitting outcomes for this particular example, it becomes evident that the reconstructions exhibit a discernible bias. The observer's subjective 'choice' of structures to align with the model introduces a level of variability. In the top row of Figure 1, distinct shock-related structures (manifested as the bending of streamers) are observed, upon which the idealized Geocentric Solar Coordinates (GCS) flux rope geometry is fitted. Consequently, there is a likelihood of overestimating the CME width. Despite this bias, it is noteworthy that the overall results, including directivity and speed for event E03, are minimally impacted. However, it is acknowledged that the complexity of structure choices can lead to larger discrepancies among different observers.

This study places emphasis on deriving de-projected CME speeds based on fits conducted at two distinct time steps. For each of the three models, the initial CME longitude and latitude are manually specified, utilizing the provided locations of the CME-accompanied Solar Flares (SFs). It is important to note that these values exhibit minimal to no substantial changes post-finalization of the fitting procedure. Consequently, the final CME directivity provided by PyThea is considered to be a relatively crude estimate. The ultimate orientations of the CME in Interplanetary (IP) space and at Earth are derived solely from qualitative information extracted from animations available at http://helioweather.net/archive/ (accessed on 5 April 2023). This approach ensures a rigorous and consistent basis for evaluating the de-projected CME speeds in our analyses.
















\subsection{Results and Discussions}
The main results of the study include correlations between the intensity of geomagnetic storms and parameters of solar and interplanetary phenomena. The study also involves the 3D reconstruction of geo-effective coronal mass ejections (CMEs) using the PyThea framework. The researchers analyzed the de-projection of CME speeds and compared them with in situ data. The study provides insights into the relationship between CME properties and geomagnetic storm intensity.

The main findings of the study on the correlations between solar and interplanetary phenomena and the intensity of geomagnetic storms are as follows:

1. The study found moderate positive correlations between the plasma compression parameters at the shock interface and the intensity of geomagnetic storms, as measured by the Dst index. These correlations were similar to or slightly larger than those obtained when interplanetary magnetic cloud speeds were used instead.

2. Interestingly, the study found that the correlation between the intensity of geomagnetic storms and the |Bz| component (minimum value during the ICME duration) was weaker compared to previous studies.

3. The study also reported strong correlations between the intensity of geomagnetic storms and different components of the electric and magnetic fields. However, these correlations were not analyzed in detail in the study.

4. It is important to note that the study did not calculate uncertainty estimates for the correlation coefficients, so the results should be interpreted with caution.

In conclusion, the study found moderate positive correlations between the intensity of geomagnetic storms and the plasma compression parameters at the shock interface. These findings contribute to our understanding of the relationship between solar and interplanetary phenomena and the occurrence of geomagnetic storms.

In the study, the comparison between the observed projected CME speeds and the CME speeds derived from PyThea showed that the correlation coefficient could be improved from 0.04 (using LASCO data) to 0.34-0.55 (using different geometrical models provided by PyThea). However, it was found that fast CMEs tend to have larger speed errors when applying different CME geometry reconstruction techniques.

There are correlations between space weather parameters during intense geomagnetic storms. Intense geomagnetic storms have been found to exhibit correlations with certain space weather parameters. According to the study mentioned in the context, correlations have been observed between the intensity of geomagnetic storms (measured by the Dst index) and parameters of solar and interplanetary phenomena. These correlations include:

1. Correlations with plasma compression parameters at the shock interface: The study found a moderately positive correlation between the downstream to upstream ratio of plasma compression parameters at the shock interface and the Dst index. This indicates that the intensity of geomagnetic storms is moderately correlated with the plasma compression at the shock interface.

2. Correlations with ICME speeds: Intense geomagnetic storms have also shown correlations with interplanetary coronal mass ejection (ICME) speeds. The study found that the ICME speeds, measured by spacecraft like Wind and ACE, exhibit correlations with the Dst index.

It is important to note that these correlations were observed in a small sample of geomagnetic storms and caution should be exercised when interpreting the results. Additionally, the study did not provide uncertainty estimates for the correlation coefficients.

\subsection{Conclusions}
The study found that certain parameters of solar and interplanetary phenomena show a positive correlation with the intensity of geomagnetic storms (GS). Specifically, the combination of the speed and orientation of the magnetic obstacle (nose-like) of coronal mass ejections (CMEs) seemed to have a positive feedback on the strength of the geomagnetic storm. However, other parameters such as CME speed, magnetic field, temperature, and density jump at the shock profile showed only moderate correlations with the geomagnetic storm intensity. The study also highlighted the importance of reliable estimation of the CME speed distribution over the entire CME structure for better modeling of CME propagation and forecasting of geomagnetic storms.

The significance of 3D de-projection efforts for CME arrival and GS forecasting is that it allows for a more accurate estimation of the direction and speed of coronal mass ejections (CMEs) as they approach Earth. By de-projecting the 3D geometry of CMEs, researchers can determine the true speed and trajectory of these large-scale structures, which is crucial for forecasting their potential impacts on Earth's geomagnetic storm (GS) intensity. This information helps in predicting the arrival time and intensity of the CMEs, providing valuable insights for space weather forecasting and mitigation efforts.

Additionally, 3D de-projection efforts help in distinguishing different parts of CMEs that might hit Earth, such as the apex or flanks. These different parts can lead to different processes in Earth's atmospheric layers, and accurately determining the CME directivity and geometry aids in understanding the specific effects and potential hazards associated with each type of CME impact.

Overall, reliable estimation of the 3D speed and directivity of CMEs through de-projection efforts is essential for improving the accuracy of CME arrival and GS forecasting, enabling better preparedness and mitigation strategies for potential space weather impacts.

3D de-projection is important for forecasting the arrival of coronal mass ejections (CMEs) and predicting the intensity of geomagnetic storms (GS) because it allows for a more accurate estimation of the CME's direction, geometry, and speed. When a CME reaches Earth, different parts of it can hit different regions, leading to varying effects on the Earth's atmospheric layers. By accurately determining the CME's directivity and geometry, scientists can better understand its propagation in interplanetary space and predict its potential impacts on Earth.

Projection effects occur when observing CMEs from Earth, as they appear as line-of-sight integrated intensity enhancements projected onto the plane-of-sky of the observing instrument. This projection can distort the true shape and speed of the CME. By de-projecting the observed data and reconstructing the CME's 3D geometry, scientists can correct for these projection effects and obtain a more reliable estimate of the CME's speed and arrival time.

Accurate estimation of the CME's speed and arrival time is crucial for space weather forecasting. It allows for early prediction of potential impacts on Earth, providing valuable lead time for preparations and mitigating the effects of geomagnetic storms. Additionally, understanding the geometry and directivity of CMEs helps in determining the specific processes that occur in the Earth's atmospheric layers during different types of CME hits.

In summary, 3D de-projection is important for forecasting the arrival of CMEs and predicting the intensity of geomagnetic storms because it improves the accuracy of estimating the CME's direction, geometry, and speed, which are essential parameters for assessing their potential impacts on Earth.

Different CME geometry reconstruction techniques can have an impact on the accuracy of CME speed measurements. In particular, fast CMEs tend to be more prone to large speed errors when using these techniques. This is likely due to the complexity in choosing the coronal structures that become visible as a result of the rapidly expanding magnetic structure of the CME. Additionally, for fast halo CMEs, there may be significant deviations in the reconstruction quality due to the overlap of shock and magnetic structure components. Therefore, the accuracy of CME speed measurements can be affected by the choice of reconstruction technique, especially for fast events.

Using remote sensing image data for early warnings about potential geomagnetic storm onsets and strength has certain limitations. Here are some of the limitations:

1. Projection Effects: Remote sensing measurements from a single spacecraft are subject to projection effects. This means that the observed parameters, such as CME speed, can be influenced by the viewing angle and may not accurately represent the true values. This can lead to uncertainties in the estimation of CME properties and their impact on Earth.

2. Speed Estimations: Estimating the speed of a CME from remote sensing image data can be challenging. Different techniques and models are used to reconstruct the 3D geometry of CMEs, but they can introduce errors and uncertainties in the speed estimation. Especially for fast CMEs, the speed errors can be significant.

3. Lack of Magnetic Field Information: Remote sensing image data alone cannot provide information about the magnetic field structure of CMEs. The strength and orientation of the magnetic field within a CME play a crucial role in determining its impact on Earth's magnetosphere. Without this information, it is difficult to accurately predict the strength of the associated geomagnetic storm.

4. Limited Coverage: Remote sensing instruments are typically located on specific spacecraft, which may have limited coverage of the Sun-Earth space. This can result in gaps in the observations and make it challenging to track and predict the propagation of CMEs and their interaction with the interplanetary medium.

5. Uncertainties in Arrival Time Prediction: Predicting the arrival time of a CME at Earth based solely on remote sensing image data can be challenging. Factors such as the interaction with the solar wind and the presence of other solar or interplanetary disturbances can affect the propagation speed and trajectory of CMEs, leading to uncertainties in arrival time predictions.

It is important to consider these limitations when using remote sensing image data for early warnings about potential geomagnetic storm onsets and strength. Integrating multiple data sources and models can help improve the accuracy of predictions and provide a more comprehensive understanding of the space weather conditions.

The study's findings have potential implications for space weather forecasting by improving the accuracy of predicting geomagnetic storms (GS) and their intensity. By analyzing the correlations between the intensity of GS and parameters of solar and interplanetary phenomena, the study aims to derive reliable solar or near-Sun parameters that can be used for early warnings about potential GS onsets and strength. The use of newly developed tools for CME de-projection, such as the PyThea software package, allows for the reconstruction of the 3D structure of coronal mass ejections (CMEs) and shock waves, which can provide more accurate information for space weather forecasting. Additionally, the study highlights the importance of considering different CME geometry reconstruction techniques to improve the accuracy of predicting fast CMEs, which are particularly prone to large speed errors. Overall, the findings of this study contribute to the advancement of space weather forecasting and the ability to predict and mitigate the potential impacts of geomagnetic storms on Earth.

The study provides valuable insights into the relationship between geomagnetic storms and solar/interplanetary factors. It examines correlations between the intensity of geomagnetic storms and parameters of solar and interplanetary phenomena. The study also includes 3D geometry reconstructions of geo-effective coronal mass ejections (CMEs) and compares on-sky and de-projected parameter values. The researchers found that a combination of fast speed and frontal orientation of the magnetic structure upon its arrival at the terrestrial magnetosphere is the best indicator for the strength of a geomagnetic storm. The study emphasizes the importance of more reliable estimations of geometry, directivity, and de-projected speeds for geomagnetic storm forecasting in operational space weather schemes.

























































