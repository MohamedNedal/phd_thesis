\chapter{Summary and Future Work}
\label{chapter5}

\section{}

\section{}

\section{}

\section{}

\section{Future Work}
The research presented in this thesis establishes an important foundation and provides a precursor for future advances that can build upon the present work. Some open questions and promising areas for future investigations include:

- Additional coronal wave statistical studies using expanded event samples and new imaging datasets from Solar Orbiter and ground observatories to improve generalizability of findings.

- Incorporating 3D analytical and numerical coronal wave propagation models for more physics-based forecasting approaches. 

- Modeling mechanisms for type III radio burst onset and time profiles using particle-in-cell and MHD models. 

- Ensemble forecasting models for SEP events combining multiple machine learning algorithms trained on multi-mission data.

- Validation of data-driven models for other solar wind driven geospace extremes such as radiation belt enhancements and ionospheric storms.

- Leveraging new solar observatory missions and assimilative models within operational prediction systems for real-time space weather forecasts.

- Exploring applications of deep learning and physics-informed machine learning to additional outstanding problems in heliophysics and astrophysics.

In summary, the present work opens exciting avenues for more cross-disciplinary studies synthesizing heliophysics domain knowledge with cutting-edge data science and artificial intelligence methods. The new generation of solar, heliospheric and geospace missions will yield transformative observations to continue advancing both science understanding and predictive capabilities.

\section{Definitions and Acronyms}
Some of the key technical terms and acronyms used in this thesis are listed below:

SEP - Solar Energetic Particle

CME - Coronal Mass Ejection 

EUV - Extreme Ultraviolet

SOHO - Solar and Heliospheric Observatory

SDO - Solar Dynamics Observatory

AIA - Atmospheric Imaging Assembly 

EIT - Extreme ultraviolet Imaging Telescope

TRACE - Transition Region and Coronal Explorer

LASCO - Large Angle and Spectrometric COronagraph

GOES - Geostationary Operational Environmental Satellite

AI - Artificial Intelligence

MHD - Magnetohydrodynamics

AU - Astronomical Unit

This provides definitions of the major domain-specific terms and measurement concepts used. Additional terminology is introduced as required in the respective chapters.
