\documentclass{article}

\usepackage[english]{babel}
% Set page size and margins
% Replace `letterpaper' with `a4paper' for UK/EU standard size
\usepackage[a4paper, top=2cm, bottom=2cm, left=3cm, right=3cm, marginparwidth=1.75cm]{geometry}
\usepackage{amsmath}
\usepackage{graphicx}
\usepackage[colorlinks=true, allcolors=blue]{hyperref}


\title{From the Sun to Earth: Exploring the Multifaceds of Solar Eruptive Events}
\author{Mohamed Nedal}

\begin{document}
\maketitle



\section{Chapter 1: Introduction}
\subsection{Background and significance of studying heliophysics}
\subsection{Overview of the solar corona and its dynamics}
\subsection{Introduction to Coronal Bright Fronts (CBFs)}
\subsection{Solar Energetic Particles (SEPs) and their relevance to space weather}
\subsection{Solar type III radio bursts and their significance}
\subsection{Objective of the thesis and organization of the chapters}

\section{Chapter 2: Characterization of the Early Dynamics of Coronal Bright Fronts}
\subsection{Overview of the methodology used for analyzing CBFs}
\subsection{Description of the Solar Particle Radiation Environment Analysis and Forecasting - Acceleration and Scattering Transport (SPREAdFAST) framework}
\subsection{Data sources and instrumentation}
\begin{enumerate}
    \item Analysis of the temporal evolution and plasma properties of CBFs.
    \item Correlation of CBF observations with SEP events near Earth.
    \item Statistical relations and distributions of shocks and plasma parameters associated with CBFs.
\end{enumerate}
\subsection{Multi-event study of early-stage SEP acceleration by CME-driven shocks}
\subsection{Multi-scale image preprocessing and feature tracking method for remote coronal waves characterization}
\subsection{Implications of the findings for space weather forecasting and SEP events}

\section{Chapter 3: Forecasting Solar Energetic Proton Integral Fluxes with Bi-Directional Long Short-Term Memory Neural Networks}
\subsection{Overview of the SEP forecasting problem and its importance}
\subsection{Introduction to the BiLSTM neural network model}
\begin{enumerate}
    \item Selection of input parameters and their relevance to SEP flux prediction.
    \item Training the BiLSTM models for different forecast windows and GOES channels.
    \item Validation and benchmarking of the SEP forecasting models.
    \item Comparison with other types of SEP prediction models.
\end{enumerate}
\subsection{Discussion of the results and implications for operational space weather forecasting}

\section{Chapter 4: Coronal Diagnostics of Solar Type-III Radio Bursts Using LOFAR and PSP Observations}
\subsection{Description of the observational data from LOFAR and PSP}
\subsection{Data preprocessing techniques}
\subsection{Characterizing the type III radio bursts}
\subsection{Imaging of radio emission sources using LOFAR interferometric observations}
\subsection{Analysis and modeling of plasma parameters and magnetic field}
\subsection{Notable findings and their implications for understanding type III radio bursts}

%\section{Chapter 5: Summary of Co-authored Papers}
%Key findings from each co-authored paper and their relevance to the field.
%\subsection{Overview of the review of solar energetic particle models}
%\subsection{Parameter study of geomagnetic storms and associated phenomena}
%\subsection{Neural network prediction of topside electron content over the Euro-African sector}

\section{Chapter 6: Conclusion and Future Directions}
\begin{enumerate}
    \item Summary of the main findings and contributions of the thesis.
    \item Limitations and potential areas for future research.
    \item Limitations and potential areas for future research.
    \item Importance of the research for advancing our understanding of heliophysics and space weather forecasting.
    \item Final remarks and closing thoughts.
\end{enumerate}


\bibliographystyle{alpha}
\bibliography{refs}

\end{document}